\documentclass[12pt,a4paper]{report}

% ----------------------- PREAMBLES -------------------------
% ==============================================================================================
% PACKAGE IMPORTS
% ----------------------------------------------------------------------------------------------
\usepackage[margin=1in]{geometry}  % Set page margins
\usepackage{graphicx}   % Enable including images
\usepackage{titlesec}  % Required for custom section titles
\usepackage{listings}   % For including code snippits
\usepackage{xcolor}      % load the package for colors
% ===============================================================================================

% ==============================================================================================
% CONFIGURATIONS
% ----------------------------------------------------------------------------------------------
% -------------------------- LISTING CONFIGURATIONS ------------------
% Custom Listing Style
\lstset{
    basicstyle=\ttfamily\normalsize, % Monospaced font
    commentstyle=\color{green!60!black}\itshape,  % Comments in green italics
    keywordstyle=\color{blue}, % Keywords in blue
    stringstyle=\color{red}, % Strings in red
    numbers=left,  % Line numbers
    numberstyle=\scriptsize,
    stepnumber=1,
    numbersep=5pt,
    frame=single,  % Box around code
    breaklines=true, % Automatic line breaks
    showstringspaces=false,
    tabsize=4,
    captionpos=b,  % Captions below code
    abovecaptionskip=0pt,
    belowskip=5pt
}

% % Change the listing name
% \renewcommand{\lstlistingname}{Code} % Changes "Listing" to "Code"

% ------------------ GRAPHICS CONFIGURATIONS ------------------------
\graphicspath{{../images/}}  % Default folder(s) for images

% -------------------- PARAGRAPH FORMATTING --------------------------
\setlength{\parindent}{0cm}      % Size of paragraph indentation
\setlength{\parskip}{12pt}       % Vertical space between paragraphs

% ------------------ SECTION FORMATTING -----------------------------
% Format section titles to be ALL CAPS
\titleformat{\section}[block]
  {\normalfont\bfseries}
  {\thesection}{0.5cm}{\MakeUppercase}

% Customize section title spacing
\titlespacing{\section}{0pt}{5pt}{0pt}

% Suppress numbering globally for sections
\setcounter{secnumdepth}{0}
% ===============================================================================================

% --------------------- LAB INFORMATION ---------------------------
\newcommand{\PracticalDate}{18th Janaury 2026}
\newcommand{\SubmissionDate}{23rd Janaury 2026}

\newcommand{\LabNo}{05}
\newcommand{\LabTitle}{Searching Algorithms}

% --------------------- LAB REPORT --------------------------------
\begin{document}
% Cover Page
% ==============================================================================================
% COVER PAGE INFORMATION (MACROS)
% ----------------------------------------------------------------------------------------------
% -------------- COLLEGE INFORMATION ------------------
\newcommand{\CollegeName}{Cosmos College of \\[12pt] Management \& Technology}
\newcommand{\Affiliation}{(Affiliated to Pokhara University)}
\newcommand{\CollegeAddress}{Sitapaila, Kathmandu}

% -------------- STUDENT INFORMATION ----------------------
\newcommand{\StudentName}{Kushal Prasad Joshi}
\newcommand{\RollNo}{230345}
\newcommand{\Faculty}{Science and Technology}
\newcommand{\Department}{ICT}
\newcommand{\Group}{B}

% ----------------- LAB INFORMATION -------------------
\newcommand{\Course}{Artifical Intelligence (CMP 346)}
\newcommand{\LabInstructor}{Mr.\ Ranjan Raj Aryal}
% ==============================================================================================

% ==============================================================================================
% COVER PAGE FORMAT
% ----------------------------------------------------------------------------------------------
\begin{titlepage}
    \centering
    \vspace*{\fill}

    % ---------------------- COLLEGE INFO ------------------------
    {\Huge \textbf{\CollegeName}}\\[12pt]
    {\huge \Affiliation}\\[12pt]
    {\LARGE \CollegeAddress}
    \vspace{2.5cm}

    % ------------------------ COLLEGE LOGO ------------------------
    \includegraphics[width=0.3\textwidth]{cosmos-logo.png}
    \vspace{2.5cm}

    % ------------------------ LAB INFO -------------------------------
    {\Large \textbf{\Course\ Lab Report}}\\[10pt]
    \textbf{LAB REPORT NO:} \LabNo\\[6pt]
    \textbf{LAB REPORT ON:}\\ \MakeUppercase{\LabTitle}
    \vspace{2.5cm}

    % ----------------------------- SUBMISSION TABLE -------------------
    \begin{tabular}{p{8cm} p{8cm}}
        \textbf{SUBMITTED BY:-} & \textbf{SUBMITTED TO:-}\\[10pt]
        \textbf{Name:} \StudentName& \textbf{Department:} \Department\\[6pt]
        \textbf{Roll No:} \RollNo& \textbf{Lab Instructor:} \LabInstructor\\[6pt]
        \textbf{Faculty: }\Faculty& \textbf{Date of Experiment:} \PracticalDate\\[6pt]
        \textbf{Group:} \Group& \textbf{Date of Submission:} \SubmissionDate\\
    \end{tabular}   
\end{titlepage}

% Title
\section{Title}
\MakeUppercase{\LabTitle}

% Define Objectives
\section{Objectives}
\begin{enumerate}
    \item To implement breadth-first search algorithm using Python.
    \item To implement depth-first search algorithm using Python.
    \item To sove water jug problem using Python.
\end{enumerate}

% Define Requirements
\section{Requirements}
\begin{itemize}
    \item Python
    \item VS Code
\end{itemize}

% Theory
\section{Theory}
\textbf{BFS:}
Breadth-First Search (BFS) is an algorithm for traversing or searching tree or graph data 
structures. It starts at the tree root (or an arbitrary node of a graph, sometimes referred 
to as a `search key') and explores the neighbor nodes at the present depth prior to moving
on to the nodes at the next depth level.

\textbf{DFS:}
Depth-First Search (DFS) is an algorithm for traversing or searching tree or graph data 
structures. The algorithm starts at the root node (selecting some arbitrary node as the
root node in the case of a graph) and explores as far as possible along each branch before 
backtracking.

\textbf{Water Jug Problem:}
The water jug problem is a classic problem in artificial intelligence and computer science.
The problem involves two jugs with different capacities and the goal is to measure out a
specific amount of water using these jugs. The problem can be solved using various search
algorithms, such as BFS and DFS.

\section{Implementation}

% BFS Implementation
\begin{lstlisting}[language=Python,  caption={Breadth-First Search Implementation}, label=listing:bfs]
from collections import deque

def bfs(graph, start):
    visited = set()
    queue = deque([start])

    print("BFS Traversal Order:", end=' ')
    
    while queue:
        node = queue.popleft()
        if node not in visited:
            print(node, end=' ')
            visited.add(node)
            queue.extend(neighbor for neighbor in graph[node] if neighbor not in visited)

# Example usage:
if __name__ == "__main__":
    graph = {
        'A' : ['B','C'],
        'B' : ['D', 'E'],
        'C' : ['F'],
        'D' : [],
        'E' : ['F'],
        'F' : []
    }

    bfs(graph, 'A')
\end{lstlisting}

% Output of BFS
\begin{lstlisting}[caption={Output of BFS Implementation}]
BFS Traversal Order: A B C D E F 
\end{lstlisting}

% DFS Implementation
\begin{lstlisting}[language=Python,  caption={Depth-First Search Implementation},label=listing:dfs]
def dfs(graph, start):
    visited = set()
    stack = [start]

    print("DFS Traversal Order:", end=' ')

    while stack:
        node = stack.pop()
        if node not in visited:
            print(node, end=' ')
            visited.add(node)
            stack.extend(reversed([neighbor for neighbor in graph[node] if neighbor not in visited]))

# Example usage:
if __name__ == "__main__":
    graph = {
        'A' : ['B','C'],
        'B' : ['D', 'E'],
        'C' : ['F'],
        'D' : [],
        'E' : ['F'],
        'F' : []
    }

    dfs(graph, 'A')
\end{lstlisting}

% Output of DFS
\begin{lstlisting}[caption={Output of DFS Implementation}]
DFS Traversal Order: A B D E F C 
\end{lstlisting}

% Water Jug Problem Implementation
\begin{lstlisting}[language=Python,  caption={Water Jug Problem Implementation},label=listing:water-jug]
###################### WATER JUG PROBLEM ######################
# Problem Statement:
    # Two jugs with capacities A and B
    # Unlimited water supply
    # Measure exactly T liters
    # Allowed operations:
        # Fill a jug
        # Empty a jug
        # Pour water from one jug to another

from collections import deque

def water_jug_problem(capacity_a, capacity_b, target):
    # Set to keep track of already visited states
    # Each state is represented as (water_in_jug_A, water_in_jug_B)
    visited = set()

    # Queue for BFS
    # Each element contains: (current_state_A, current_state_B, path_taken)
    queue = deque()

    # Initial state: both jugs are empty
    queue.append((0, 0, []))

    # Continue BFS until queue is empty
    while queue:
        # Remove the front state from the queue
        a, b, path = queue.popleft()

        # If the target amount is found in either jug
        if a == target or b == target:
            # Add final state to path and return solution
            path.append((a, b))
            return path

        # Skip state if already visited
        if (a, b) in visited:
            continue

        # Mark current state as visited
        visited.add((a, b))

        # Add current state to the path
        path = path + [(a, b)]

        # Generate all possible next states using allowed operations
        next_states = [

            # Fill Jug A completely
            (capacity_a, b),

            # Fill Jug B completely
            (a, capacity_b),

            # Empty Jug A
            (0, b),

            # Empty Jug B
            (a, 0),

            # Pour water from Jug A to Jug B
            # Amount poured is the minimum of:
            #   - water available in A
            #   - remaining capacity in B
            (a - min(a, capacity_b - b),
             b + min(a, capacity_b - b)),

            # Pour water from Jug B to Jug A
            (a + min(b, capacity_a - a),
             b - min(b, capacity_a - a))
        ]

        # Add all unvisited next states to the BFS queue
        for state in next_states:
            if state not in visited:
                queue.append((state[0], state[1], path))

    # If BFS completes without finding target
    return None

# Example usage:
if __name__ == "__main__":
    solution = water_jug_problem(
        capacity_a = int(input("Enter capacity of Jug A: ")),
        capacity_b = int(input("Enter capacity of Jug B: ")),
        target = int(input("Enter target amount of water: "))
    )

    print("Steps to reach target:", end=' ')
    if solution:
        for step in solution:
            print(step, end=' ')
    else:
        print("No solution exists.")
\end{lstlisting}

% Output of Water Jug Problem
\begin{lstlisting}[caption={Output of Water Jug Problem Implementation}]
Enter capacity of Jug A: 4
Enter capacity of Jug B: 3
Enter target amount of water: 2
Steps to reach target: (0, 0) (0, 3) (3, 0) (3, 3) (4, 2) 
\end{lstlisting}

% Results and Conclusion
\section{Results and Conclusion}
In this lab, we successfully implemented the Breadth-First Search (BFS) and Depth-First Search (DFS) 
algorithms in Python. We also solved the Water Jug Problem using the BFS approach. The implementations 
were tested with example graphs and jug capacities, yielding correct traversal orders and solution steps.
\end{document}