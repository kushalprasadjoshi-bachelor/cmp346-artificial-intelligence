\documentclass[12pt,a4paper]{report}

% ----------------------- PREAMBLES -------------------------
% ==============================================================================================
% PACKAGE IMPORTS
% ----------------------------------------------------------------------------------------------
\usepackage[margin=1in]{geometry}  % Set page margins
\usepackage{graphicx}   % Enable including images
\usepackage{titlesec}  % Required for custom section titles
\usepackage{listings}   % For including code snippits
\usepackage{xcolor}      % load the package for colors
% ===============================================================================================

% ==============================================================================================
% CONFIGURATIONS
% ----------------------------------------------------------------------------------------------
% -------------------------- LISTING CONFIGURATIONS ------------------
% Custom Listing Style
\lstset{
    basicstyle=\ttfamily\normalsize, % Monospaced font
    commentstyle=\color{green!60!black}\itshape,  % Comments in green italics
    keywordstyle=\color{blue}, % Keywords in blue
    stringstyle=\color{red}, % Strings in red
    numbers=left,  % Line numbers
    numberstyle=\scriptsize,
    stepnumber=1,
    numbersep=5pt,
    frame=single,  % Box around code
    breaklines=true, % Automatic line breaks
    showstringspaces=false,
    tabsize=4,
    captionpos=b,  % Captions below code
    abovecaptionskip=0pt,
    belowskip=5pt
}

% % Change the listing name
% \renewcommand{\lstlistingname}{Code} % Changes "Listing" to "Code"

% ------------------ GRAPHICS CONFIGURATIONS ------------------------
\graphicspath{{../images/}}  % Default folder(s) for images

% -------------------- PARAGRAPH FORMATTING --------------------------
\setlength{\parindent}{0cm}      % Size of paragraph indentation
\setlength{\parskip}{12pt}       % Vertical space between paragraphs

% ------------------ SECTION FORMATTING -----------------------------
% Format section titles to be ALL CAPS
\titleformat{\section}[block]
  {\normalfont\bfseries}
  {\thesection}{0.5cm}{\MakeUppercase}

% Customize section title spacing
\titlespacing{\section}{0pt}{5pt}{0pt}

% Suppress numbering globally for sections
\setcounter{secnumdepth}{0}
% ===============================================================================================

% --------------------- LAB INFORMATION ---------------------------
\newcommand{\PracticalDate}{15th December 2025}
\newcommand{\SubmissionDate}{21st December 2025}

\newcommand{\LabNo}{01}
\newcommand{\LabTitle}{Introduction to Prolog}

% --------------------- LAB REPORT --------------------------------
\begin{document}
% Cover Page
% ==============================================================================================
% COVER PAGE INFORMATION (MACROS)
% ----------------------------------------------------------------------------------------------
% -------------- COLLEGE INFORMATION ------------------
\newcommand{\CollegeName}{Cosmos College of \\[12pt] Management \& Technology}
\newcommand{\Affiliation}{(Affiliated to Pokhara University)}
\newcommand{\CollegeAddress}{Sitapaila, Kathmandu}

% -------------- STUDENT INFORMATION ----------------------
\newcommand{\StudentName}{Kushal Prasad Joshi}
\newcommand{\RollNo}{230345}
\newcommand{\Faculty}{Science and Technology}
\newcommand{\Department}{ICT}
\newcommand{\Group}{B}

% ----------------- LAB INFORMATION -------------------
\newcommand{\Course}{Artifical Intelligence (CMP 346)}
\newcommand{\LabInstructor}{Mr.\ Ranjan Raj Aryal}
% ==============================================================================================

% ==============================================================================================
% COVER PAGE FORMAT
% ----------------------------------------------------------------------------------------------
\begin{titlepage}
    \centering
    \vspace*{\fill}

    % ---------------------- COLLEGE INFO ------------------------
    {\Huge \textbf{\CollegeName}}\\[12pt]
    {\huge \Affiliation}\\[12pt]
    {\LARGE \CollegeAddress}
    \vspace{2.5cm}

    % ------------------------ COLLEGE LOGO ------------------------
    \includegraphics[width=0.3\textwidth]{cosmos-logo.png}
    \vspace{2.5cm}

    % ------------------------ LAB INFO -------------------------------
    {\Large \textbf{\Course\ Lab Report}}\\[10pt]
    \textbf{LAB REPORT NO:} \LabNo\\[6pt]
    \textbf{LAB REPORT ON:}\\ \MakeUppercase{\LabTitle}
    \vspace{2.5cm}

    % ----------------------------- SUBMISSION TABLE -------------------
    \begin{tabular}{p{8cm} p{8cm}}
        \textbf{SUBMITTED BY:-} & \textbf{SUBMITTED TO:-}\\[10pt]
        \textbf{Name:} \StudentName& \textbf{Department:} \Department\\[6pt]
        \textbf{Roll No:} \RollNo& \textbf{Lab Instructor:} \LabInstructor\\[6pt]
        \textbf{Faculty: }\Faculty& \textbf{Date of Experiment:} \PracticalDate\\[6pt]
        \textbf{Group:} \Group& \textbf{Date of Submission:} \SubmissionDate\\
    \end{tabular}   
\end{titlepage}

% Title
\section{Title}
\MakeUppercase{\LabTitle}

% Define Objectives
\section{Objectives}
\begin{enumerate}
    \item To understand the basic working of the SWI-Prolog environment.
    \item To write and execute simple Prolog programs.
\end{enumerate}

% Define Requirements
\section{Requirements}
\begin{itemize}
    \item SWI-prolog
    \item Operating System: Windows
    \item Text Edioter / SWI-Prolog Edioter
\end{itemize}

% Theory
\section{Theory}
Prolog is logic programming language based on formal logic. Instead of executing instructions 
sequentially, Prolog programs define facts and rules, and computation is performed by 
querying these definations.

SWI-Prolog is a widely used Prolog implementation that provides an interactive environment, 
compiler, and extensive libraries.

% Procedure to Run Programs/Queries
\section{Procedure}
\begin{enumerate}
    \item Install and open the SWI-Prolog environment on the computer.
    \item Create a new Prolog source file with the \texttt{.pl} extension.
    \item Write the required Prolog predicates (facts and rules) in the source file.
    \item Save the program in an appropriate working directory.
    \item Open the Prolog terminal and change the directory to the location of the program file.
    \item Load the Prolog source file into the interpreter using the \texttt{consult\!('file\_name.pl')} command.
    \item Execute the required queries or predicates to observe the output.
    \item Verify the output and repeat execution if necessary.
\end{enumerate}

% Lab Exercises/Implementations
\section{Implementation}
% Program to display "Hello World" using SWI-Prolog
\begin{lstlisting}[language=Prolog,  caption={Prolog Program to Display Hello World}, label=listing:hello]
hello:-
    format('Hello World').
\end{lstlisting}

\begin{lstlisting}[caption={Output of the Hello World Prolog Program}]
?- hello.
Hello World
true.
\end{lstlisting}

% Solve the problems using SWI prolog as per the goals given
\begin{lstlisting}[language=Prolog,  caption={Prolog Facts and Rules for Weekly Class Schedule and Subject Difficulty}, label=listing:subject-info]
% ======== FACTS ==========
% schedule(day, subject)
schedule(monday,programming).
schedule(tuesday,math).
schedule(tuesday,english).
schedule(wednesday,programming).
schedule(wednesday,spanish).
schedule(thursday,circuits).
schedule(friday,none).

% difficulty(subject, level)
difficulty(programming,hard).
difficulty(math,hard).
difficulty(english,easy).
difficulty(spanish,medium).
difficulty(circuits,hard).

% ================ RULES ===========
classinformation(Day,Class,Diff) :-
    schedule(Day,Class),
    difficulty(Class,Diff).
\end{lstlisting}

\begin{lstlisting}[caption={Output of Queries on Class Schedule and Difficulty Information}]
?- schedule(monday,programming). 
true.

?- schedule(monday, english). 
false.

?- difficulty( programming, easy). 
false.

?- difficulty(programming, hard). 
true.

?- classinformation(tuesday, Class, Diff). 
Class = math,
Diff = hard ;
Class = english,
Diff = easy.

?- classinformation(tuesday, Class, easy). 
Class = english.

?- classinformation(Day, Class, hard). 
Day = monday,
Class = programming ;
Day = tuesday,
Class = math ;
Day = wednesday,
Class = programming ;
Day = thursday,
Class = circuits ;
false.

?- classinformation(Day, english, hard).
false.

?- classinformation(Day, english, easy). 
Day = tuesday.
\end{lstlisting}

% Solve the problems using SWI prolog as per the goals given
\begin{lstlisting}[language=Prolog,  caption={Prolog Facts and Rules for Restaurant Information System}, label=listing:restaurant-info]
% ============ FACTS =================
% Restaurant facts: cuisine(type, name)
cuisine(italian, marios).
cuisine(italian, pasta_paradise).
cuisine(mexican, taco_fiesta).
cuisine(mexican, chili_beans).
cuisine(indian, spice_route).
cuisine(japanese, sushi_zen).
cuisine(japanese, wasabi_world).
cuisine(burger, burger_kingdom).

% Price level facts: price(restaurant, level)
price(marios, expensive).
price(pasta_paradise, moderate).
price(taco_fiesta, cheap).
price(chili_beans, moderate).
price(spice_route, expensive).
price(sushi_zen, expensive).
price(wasabi_world, moderate).
price(burger_kingdom, cheap).

% Location facts: location(restaurant, area)
location(marios, downtown).
location(pasta_paradise, uptown).
location(taco_fiesta, downtown).
location(chili_beans, midtown).
location(spice_route, downtown).
location(sushi_zen, uptown).
location(wasabi_world, downtown).
location(burger_kingdom, midtown).

% Rating facts: rating(restaurant, stars)
rating(marios, 4).
rating(pasta_paradise, 3).
rating(taco_fiesta, 5).
rating(chili_beans, 4).
rating(spice_route, 5).
rating(sushi_zen, 4).
rating(wasabi_world, 3).
rating(burger_kingdom, 4).

% ================= RULES ==================
% Combined information rule
restaurantinfo(Name, Cuisine, Price, Area, Stars) :-
    cuisine(Cuisine, Name),
    price(Name, Price),
    location(Name, Area),
    rating(Name, Stars).
\end{lstlisting}

\begin{lstlisting}[caption={Output of Queries on Restaurant Information System}]
?- cuisine(italian, marios). 
true.

?- cuisine(chinese, spice_route). 
false.

?- price(sushi_zen, expensive). 
true.

?- price(taco_fiesta, expensive).
false.

?- location(pasta_paradise, downtown). 
false.

?- cuisine(italian, Restaurant).
Restaurant = marios ;
Restaurant = pasta_paradise.

?- price(Restaurant, cheap). 
Restaurant = taco_fiesta ;
Restaurant = burger_kingdom.

?- location(Restaurant, downtown). 
Restaurant = marios ;
Restaurant = taco_fiesta ;
Restaurant = spice_route ;
Restaurant = wasabi_world.

?- rating(Restaurant, 5).
Restaurant = taco_fiesta ;
Restaurant = spice_route.

?-  restaurantinfo(Name, mexican, Price, Area, Stars). 
Name = taco_fiesta,
Price = cheap,
Area = downtown,
Stars = 5 ;
Name = chili_beans,
Price = moderate,
Area = midtown,
Stars = 4.

?-  restaurantinfo(Name, Cuisine, expensive, downtown, Stars).
Name = marios,
Cuisine = italian,
Stars = 4 ;
Name = spice_route,
Cuisine = indian,
Stars = 5 ;
false.

?-  restaurantinfo(Name, japanese, Price, Area, 4). 
Name = sushi_zen,
Price = expensive,
Area = uptown ;
false.

?- restaurantinfo(taco_fiesta, Cuisine, Price, Area, Stars).
Cuisine = mexican,
Price = cheap,
Area = downtown,
Stars = 5.

?-  cuisine(Cuisine, Name), price(Name, cheap). 
Cuisine = mexican,
Name = taco_fiesta ;
Cuisine = burger,
Name = burger_kingdom.

?- restaurantinfo(Name, Cuisine, moderate, downtown, Rating), Rating > 3.
false.

?- location(Name, uptown), rating(Name, Stars), Stars >= 4.
Name = sushi_zen,
Stars = 4.

?- restaurantinfo(Name, italian, Price, Area, Stars), (Price = cheap ; Price = moderate). 
Name = pasta_paradise,
Price = moderate,
Area = uptown,
Stars = 3.

?-  restaurantinfo(Name, _, expensive, _, 5). 
Name = spice_route ;
false.

?- restaurantinfo(Name, Cuisine, _, downtown, _), Cuisine \= italian. 
Name = taco_fiesta,
Cuisine = mexican ;
Name = spice_route,
Cuisine = indian ;
Name = wasabi_world,
Cuisine = japanese ;
false.

?- restaurantinfo(_, mexican, cheap, _, 4).
false.
\end{lstlisting}

% Result and Conclusion From the Lab Session
\section{Result and Conclusion}
The Prolog programs were executed successfully in the SWI-Prolog environment, and all facts, 
rules, and predicates produced the expected outputs. The results confirmed correct knowledge 
representation and logical inference through queries. This lab helped in understanding the 
basics of Prolog programming, the use of facts and rules, and effective interaction with the 
SWI-Prolog environment, highlighting the declarative approach of logic programming.

\end{document}
