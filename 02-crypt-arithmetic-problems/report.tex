\documentclass[12pt,a4paper]{report}

% ------------------ PREAMBLES -----------------------
% ==============================================================================================
% PACKAGE IMPORTS
% ----------------------------------------------------------------------------------------------
\usepackage[margin=1in]{geometry}  % Set page margins
\usepackage{graphicx}   % Enable including images
\usepackage{titlesec}  % Required for custom section titles
\usepackage{listings}   % For including code snippits
\usepackage{xcolor}      % load the package for colors
% ===============================================================================================

% ==============================================================================================
% CONFIGURATIONS
% ----------------------------------------------------------------------------------------------
% -------------------------- LISTING CONFIGURATIONS ------------------
% Custom Listing Style
\lstset{
    basicstyle=\ttfamily\normalsize, % Monospaced font
    commentstyle=\color{green!60!black}\itshape,  % Comments in green italics
    keywordstyle=\color{blue}, % Keywords in blue
    stringstyle=\color{red}, % Strings in red
    numbers=left,  % Line numbers
    numberstyle=\scriptsize,
    stepnumber=1,
    numbersep=5pt,
    frame=single,  % Box around code
    breaklines=true, % Automatic line breaks
    showstringspaces=false,
    tabsize=4,
    captionpos=b,  % Captions below code
    abovecaptionskip=0pt,
    belowskip=5pt
}

% % Change the listing name
% \renewcommand{\lstlistingname}{Code} % Changes "Listing" to "Code"

% ------------------ GRAPHICS CONFIGURATIONS ------------------------
\graphicspath{{../images/}}  % Default folder(s) for images

% -------------------- PARAGRAPH FORMATTING --------------------------
\setlength{\parindent}{0cm}      % Size of paragraph indentation
\setlength{\parskip}{12pt}       % Vertical space between paragraphs

% ------------------ SECTION FORMATTING -----------------------------
% Format section titles to be ALL CAPS
\titleformat{\section}[block]
  {\normalfont\bfseries}
  {\thesection}{0.5cm}{\MakeUppercase}

% Customize section title spacing
\titlespacing{\section}{0pt}{5pt}{0pt}

% Suppress numbering globally for sections
\setcounter{secnumdepth}{0}
% ===============================================================================================

% --------------------- LAB INFORMATION ---------------------------
\newcommand{\PracticalDate}{21st December 2025}
\newcommand{\SubmissionDate}{28th December 2025}

\newcommand{\LabNo}{02}
\newcommand{\LabTitle}{Crypt Arithmetic Problems}

% --------------------- LAB REPORT --------------------------------
\begin{document}
% Cover Page
% ==============================================================================================
% COVER PAGE INFORMATION (MACROS)
% ----------------------------------------------------------------------------------------------
% -------------- COLLEGE INFORMATION ------------------
\newcommand{\CollegeName}{Cosmos College of \\[12pt] Management \& Technology}
\newcommand{\Affiliation}{(Affiliated to Pokhara University)}
\newcommand{\CollegeAddress}{Sitapaila, Kathmandu}

% -------------- STUDENT INFORMATION ----------------------
\newcommand{\StudentName}{Kushal Prasad Joshi}
\newcommand{\RollNo}{230345}
\newcommand{\Faculty}{Science and Technology}
\newcommand{\Department}{ICT}
\newcommand{\Group}{B}

% ----------------- LAB INFORMATION -------------------
\newcommand{\Course}{Artifical Intelligence (CMP 346)}
\newcommand{\LabInstructor}{Mr.\ Ranjan Raj Aryal}
% ==============================================================================================

% ==============================================================================================
% COVER PAGE FORMAT
% ----------------------------------------------------------------------------------------------
\begin{titlepage}
    \centering
    \vspace*{\fill}

    % ---------------------- COLLEGE INFO ------------------------
    {\Huge \textbf{\CollegeName}}\\[12pt]
    {\huge \Affiliation}\\[12pt]
    {\LARGE \CollegeAddress}
    \vspace{2.5cm}

    % ------------------------ COLLEGE LOGO ------------------------
    \includegraphics[width=0.3\textwidth]{cosmos-logo.png}
    \vspace{2.5cm}

    % ------------------------ LAB INFO -------------------------------
    {\Large \textbf{\Course\ Lab Report}}\\[10pt]
    \textbf{LAB REPORT NO:} \LabNo\\[6pt]
    \textbf{LAB REPORT ON:}\\ \MakeUppercase{\LabTitle}
    \vspace{2.5cm}

    % ----------------------------- SUBMISSION TABLE -------------------
    \begin{tabular}{p{8cm} p{8cm}}
        \textbf{SUBMITTED BY:-} & \textbf{SUBMITTED TO:-}\\[10pt]
        \textbf{Name:} \StudentName& \textbf{Department:} \Department\\[6pt]
        \textbf{Roll No:} \RollNo& \textbf{Lab Instructor:} \LabInstructor\\[6pt]
        \textbf{Faculty: }\Faculty& \textbf{Date of Experiment:} \PracticalDate\\[6pt]
        \textbf{Group:} \Group& \textbf{Date of Submission:} \SubmissionDate\\
    \end{tabular}   
\end{titlepage}

% Title
\section{Title}
\MakeUppercase{\LabTitle{}}

% Define Objectives
\section{Objectives}
\begin{enumerate}
    \item To understand Crypt Arithmetic Problems using Prolog.
    \item To solve the problem for the following arithmetic problems:
    \begin{itemize}
        \item LETS + WAVE = LATER  
        \item SEND + MORE =  MONEY 
        \item TWO + TWO = FOUR 
        \item IT + IS = ME 
        \item BASE + BALL = GAMES 
    \end{itemize}
\end{enumerate}

% Define Requirements
\section{Requirements}
\begin{itemize}
    \item SWI-prolog
    \item Operating System: Windows
    \item Text Edioter / SWI-Prolog Edioter
\end{itemize}

% Theory
\section{Theory}
Crypt-arithmetic problems are a class of mathematical puzzles where letters are used to represent 
digits in an arithmetic equation. Each letter corresponds to a unique digit from 0 to 9, and no 
two letters share the same digit. The goal is to find the digit assignment that makes the 
equation numerically correct, with the additional constraint that leading letters cannot be zero. 
These problems are commonly used to demonstrate logical reasoning, constraint satisfaction, and 
search techniques in artificial intelligence.

% Lab Exercises/Implementations
\section{Implementation}

% LETS + WAVE = LATER  
\begin{lstlisting}[language=Prolog,  caption={LET'S + WAVE = LATER}, label=listing:lets_wave_later]
% CRYPTO-ARITHMETIC PROBLEM: LETS + WAVE = LATER
% Each letter represents a unique digit (0-9)
% No leading zeros (L and W cannot be 0)

solution(Z) :-               % Main predicate to find solution
    digit(L), L>0,           % Get digit for L, L must be >0 (no leading zero)
    digit(E),               % Get digit for E
    digit(T),               % Get digit for T
    digit(S),               % Get digit for S
    digit(W), W>0,          % Get digit for W, W must be >0 (no leading zero)
    digit(A),               % Get digit for A
    digit(V),               % Get digit for V
    digit(E),               % Get digit for E (same E as above)
    digit(R),               % Get digit for R

    % Convert LETS to number: 1000*L + 100*E + 10*T + S
    % Convert WAVE to number: 1000*W + 100*A + 10*V + E
    % Convert LATER to number: 10000*L + 1000*A + 100*T + 10*E + R
    % Equation: LETS + WAVE = LATER
    1000*L + 100*E + 10*T + S + 1000*W + 100*A + 10*V + E =:=
    10000*L + 1000*A + 100*T + 10*E + R,

    Z = [L,E,T,S,W,A,V,R],  % Store all digits in list Z
    different(Z).           % Check that all digits are unique

% DIGIT FACTS: Defines possible digits (0 through 9)
digit(0).                   % Digit can be 0
digit(1).                   % Digit can be 1
digit(2).                   % Digit can be 2
digit(3).                   % Digit can be 3
digit(4).                   % Digit can be 4
digit(5).                   % Digit can be 5
digit(6).
digit(7).
digit(8).
digit(9).
% Digit can be 6
% Digit can be 7
% Digit can be 8
% Digit can be 9
% DIFFERENT PREDICATE: Checks if all elements in a list are unique
% Base case: Empty list always has unique elements
different([]).
% Recursive case: Check if head is not in tail, then check tail recursively
different([X|P]) :-
not(member(X,P)),
different(P).
% X should not be member of the rest (P)
% Recursively check the rest of the list
\end{lstlisting}

\begin{lstlisting}[caption={Solution of LET'S + WAVE = LATER}]
?- solution(Z).
Z = [1, 5, 6, 7, 9, 0, 8, 2] ;
false.
\end{lstlisting}

% SEND + MORE =  MONEY 
\begin{lstlisting}[language=Prolog,  caption={SEND + MORE = MONEY}, label=listing:send_more_money]
solution(L):-digit(S),S>0,digit(E),digit(N),digit(D),
digit(M),M>0,digit(O),digit(R),digit(Y),
1000*S+100*E+10*N+D+1000*M+100*O+10*R+E=:=10000*M+1000*O+100*N+10*E+Y,
L=[S,E,N,D,M,O,R,Y], different(L).
digit(0).
digit(1).
digit(2).
digit(3).
digit(4).
digit(5).
digit(6).
digit(7).
digit(8).
digit(9).
different([]).
different([X|R]):-not(member(X,R)),different(R).
\end{lstlisting}

\begin{lstlisting}[caption={Solution of SEND + MORE = MONEY}]
?- solution(L).
L = [9, 5, 6, 7, 1, 0, 8, 2] ;
false.
\end{lstlisting}

% TWO + TWO = FOUR 
\begin{lstlisting}[language=Prolog,  caption={TWO + TWO = FOUR}, label=listing:two_two_four]
% TWO + TWO = FOUR
% Each letter represents a unique digit (0-9)
% T and F cannot be 0
solution(Z) :-
    digit(T), T>0,          % T cannot be 0
    digit(W),
    digit(O),
    digit(F), F>0,          % F cannot be 0
    digit(U),
    digit(R),
    % Convert TWO: 100*T + 10*W + O
    % Convert TWO again: 100*T + 10*W + O
    % Convert FOUR: 1000*F + 100*O + 10*U + R
    (100*T + 10*W + O) * 2 =:= 1000*F + 100*O + 10*U + R,
    Z = [T,W,O,F,U,R],
    different(Z).

% DIGIT FACTS: Defines possible digits (0 through 9)
digit(0).                   % Digit can be 0
digit(1).                   % Digit can be 1
digit(2).                   % Digit can be 2
digit(3).                   % Digit can be 3
digit(4).                   % Digit can be 4
digit(5).                   % Digit can be 5
digit(6).                   % Digit can be 6
digit(7).                   % Digit can be 7
digit(8).                   % Digit can be 8
digit(9).                   % Digit can be 9

% DIFFERENT PREDICATE: Checks if all elements in a list are unique

% Base case: Empty list always has unique elements
different([]).

% Recursive case: Check if head is not in tail, then check tail recursively
different([X|P]) :-
    not(member(X,P)),       % X should not be member of the rest (P)
    different(P).           % Recursively check the rest of the list
\end{lstlisting}

\begin{lstlisting}[caption={Solution of TWO + TWO = FOUR}]
?- solution(Z).
Z = [7, 3, 4, 1, 6, 8] ;
Z = [7, 6, 5, 1, 3, 0] ;
Z = [8, 3, 6, 1, 7, 2] ;
Z = [8, 4, 6, 1, 9, 2] ;
Z = [8, 6, 7, 1, 3, 4] ;
Z = [9, 2, 8, 1, 5, 6] ;
Z = [9, 3, 8, 1, 7, 6] ;
false.
\end{lstlisting}

% IT + IS = ME 
\begin{lstlisting}[language=Prolog,  caption={IT + IS = ME}, label=listing:it_is_me]
% IT + IS = ME
% Simple puzzle for beginners

solution(L) :-
    digit(I), I>0,          % I cannot be 0
    digit(T),
    digit(S),
    digit(M), M>0,          % M cannot be 0
    digit(E),
    % Convert IT: 10*I + T
    % Convert IS: 10*I + S
    % Convert ME: 10*M + E
    10*I + T + 10*I + S =:= 10*M + E,
    L = [I,T,S,M,E],
    different(L).

% DIGIT FACTS: Defines possible digits (0 through 9)
digit(0).                   % Digit can be 0
digit(1).                   % Digit can be 1
digit(2).                   % Digit can be 2
digit(3).                   % Digit can be 3
digit(4).                   % Digit can be 4
digit(5).                   % Digit can be 5
digit(6).                   % Digit can be 6
digit(7).                   % Digit can be 7
digit(8).                   % Digit can be 8
digit(9).                   % Digit can be 9

% DIFFERENT PREDICATE: Checks if all elements in a list are unique

% Base case: Empty list always has unique elements
different([]).

% Recursive case: Check if head is not in tail, then check tail recursively
different([X|P]) :-
    not(member(X,P)),       % X should not be member of the rest (P)
    different(P).           % Recursively check the rest of the list
\end{lstlisting}

\begin{lstlisting}[caption={Solution of IT + IS = ME}]
?- consult('04-it-is-me.pl').
true.

?- solution(L).
L = [1, 2, 8, 3, 0] ;
L = [1, 3, 4, 2, 7] ;
L = [1, 3, 5, 2, 8] ;
L = [1, 3, 6, 2, 9] ;
L = [1, 4, 3, 2, 7] ;
L = [1, 4, 5, 2, 9] ;
L = [1, 4, 6, 3, 0] ;
L = [1, 4, 8, 3, 2] ;
L = [1, 5, 3, 2, 8] ;
L = [1, 5, 4, 2, 9] ;
L = [1, 5, 7, 3, 2] ;
L = [1, 5, 9, 3, 4] ;
L = [1, 6, 3, 2, 9] ;
L = [1, 6, 4, 3, 0] ;
L = [1, 6, 8, 3, 4] ;
L = [1, 6, 9, 3, 5] ;
L = [1, 7, 5, 3, 2] ;
L = [1, 7, 8, 3, 5] ;
L = [1, 7, 9, 3, 6] ;
L = [1, 8, 2, 3, 0] ;
L = [1, 8, 4, 3, 2] ;
L = [1, 8, 6, 3, 4] ;
L = [1, 8, 7, 3, 5] ;
L = [1, 8, 9, 3, 7] ;
L = [1, 9, 5, 3, 4] ;
L = [1, 9, 6, 3, 5] ;
L = [1, 9, 7, 3, 6] ;
L = [1, 9, 8, 3, 7] ;
L = [2, 1, 5, 4, 6] ;
L = [2, 1, 6, 4, 7] ;
L = [2, 1, 7, 4, 8] ;
L = [2, 1, 8, 4, 9] ;
L = [2, 1, 9, 5, 0] ;
L = [2, 3, 5, 4, 8] ;
L = [2, 3, 6, 4, 9] ;
L = [2, 3, 7, 5, 0] ;
L = [2, 3, 8, 5, 1] ;
L = [2, 4, 6, 5, 0] ;
L = [2, 4, 7, 5, 1] ;
L = [2, 4, 9, 5, 3] ;
L = [2, 5, 1, 4, 6] ;
L = [2, 5, 3, 4, 8] ;
L = [2, 6, 1, 4, 7] ;
L = [2, 6, 3, 4, 9] ;
L = [2, 6, 4, 5, 0] ;
L = [2, 6, 7, 5, 3] ;
L = [2, 6, 8, 5, 4] ;
L = [2, 7, 1, 4, 8] ;
L = [2, 7, 3, 5, 0] ;
L = [2, 7, 4, 5, 1] ;
L = [2, 7, 6, 5, 3] ;
L = [2, 7, 9, 5, 6] ;
L = [2, 8, 1, 4, 9] ;
L = [2, 8, 3, 5, 1] ;
L = [2, 8, 6, 5, 4] ;
L = [2, 8, 9, 5, 7] ;
L = [2, 9, 1, 5, 0] ;
L = [2, 9, 4, 5, 3] ;
L = [2, 9, 7, 5, 6] ;
L = [2, 9, 8, 5, 7] ;
L = [3, 1, 4, 6, 5] ;
L = [3, 1, 7, 6, 8] ;
L = [3, 1, 8, 6, 9] ;
L = [3, 1, 9, 7, 0] ;
L = [3, 2, 5, 6, 7] ;
L = [3, 2, 7, 6, 9] ;
L = [3, 2, 8, 7, 0] ;
L = [3, 2, 9, 7, 1] ;
L = [3, 4, 1, 6, 5] ;
L = [3, 4, 5, 6, 9] ;
L = [3, 4, 6, 7, 0] ;
L = [3, 4, 8, 7, 2] ;
L = [3, 5, 2, 6, 7] ;
L = [3, 5, 4, 6, 9] ;
L = [3, 5, 6, 7, 1] ;
L = [3, 5, 9, 7, 4] ;
L = [3, 6, 4, 7, 0] ;
L = [3, 6, 5, 7, 1] ;
L = [3, 6, 8, 7, 4] ;
L = [3, 6, 9, 7, 5] ;
L = [3, 7, 1, 6, 8] ;
L = [3, 7, 2, 6, 9] ;
L = [3, 8, 1, 6, 9] ;
L = [3, 8, 2, 7, 0] ;
L = [3, 8, 4, 7, 2] ;
L = [3, 8, 6, 7, 4] ;
L = [3, 9, 1, 7, 0] ;
L = [3, 9, 2, 7, 1] ;
L = [3, 9, 5, 7, 4] ;
L = [3, 9, 6, 7, 5] ;
L = [4, 1, 2, 8, 3] ;
L = [4, 1, 5, 8, 6] ;
L = [4, 1, 6, 8, 7] ;
L = [4, 2, 1, 8, 3] ;
L = [4, 2, 3, 8, 5] ;
L = [4, 2, 5, 8, 7] ;
L = [4, 2, 7, 8, 9] ;
L = [4, 2, 8, 9, 0] ;
L = [4, 3, 2, 8, 5] ;
L = [4, 3, 6, 8, 9] ;
L = [4, 3, 7, 9, 0] ;
L = [4, 3, 8, 9, 1] ;
L = [4, 5, 1, 8, 6] ;
L = [4, 5, 2, 8, 7] ;
L = [4, 5, 6, 9, 1] ;
L = [4, 5, 7, 9, 2] ;
L = [4, 5, 8, 9, 3] ;
L = [4, 6, 1, 8, 7] ;
L = [4, 6, 3, 8, 9] ;
L = [4, 6, 5, 9, 1] ;
L = [4, 6, 7, 9, 3] ;
L = [4, 7, 2, 8, 9] ;
L = [4, 7, 3, 9, 0] ;
L = [4, 7, 5, 9, 2] ;
L = [4, 7, 6, 9, 3] ;
L = [4, 7, 8, 9, 5] ;
L = [4, 8, 2, 9, 0] ;
L = [4, 8, 3, 9, 1] ;
L = [4, 8, 5, 9, 3] ;
L = [4, 8, 7, 9, 5] ;
false.
\end{lstlisting}


% BASE + BALL = GAMES
\begin{lstlisting}[language=Prolog,  caption={BASE + BALL = GAMES}, label=listing:base_ball_games]
% ARGUMENT: Solution - List of digits [B,A,S,E,L,G,M]
% ------------------------------------------------------
solution(Solution) :-
    % Get digit for B (cannot be 0 as it's leading digit)
    digit(B), B > 0,
    digit(A),
    digit(S),
    digit(E),
    digit(L),
    % Get digit for G (cannot be 0 as it's leading digit)
    digit(G), G > 0,
    digit(M),

    % BASE  = 1000*B + 100*A + 10*S + E
    % BALL  = 1000*B + 100*A + 10*L + L
    % GAMES = 10000*G + 1000*A + 100*M + 10*E + S

    1000*B + 100*A + 10*S + E +
    1000*B + 100*A + 10*L + L =:=
    10000*G + 1000*A + 100*M + 10*E + S,

    % Store solution as list [B,A,S,E,L,G,M]
    Solution = [B,A,S,E,L,G,M],

    % Verify all digits are unique
    different(Solution).


% FACTS: digit/1
% DESCRIPTION: Defines possible digits 0 through 9

digit(0).  % Digit can be 0
digit(1).  % Digit can be 1
digit(2).  % Digit can be 2
digit(3).  % Digit can be 3
digit(4).  % Digit can be 4
digit(5).  % Digit can be 5
digit(6).  % Digit can be 6
digit(7).  % Digit can be 7
digit(8).  % Digit can be 8
digit(9).  % Digit can be 9


different([]).  % Base case: empty list has all unique elements

different([X|P]) :-          % Recursive case: list with head X and tail P
    not(member(X,P)),        % Check X is not a member of the tail P
    different(P).            % Recursively check the rest of the list
\end{lstlisting}

\begin{lstlisting}[caption={Solution of BASE + BALL = GAMES}]
?- solution(Solution).
Solution = [7, 4, 8, 3, 5, 1, 9] ;
false.
\end{lstlisting}

% Result and Conclusion
\section{Result and Conclusion}
The crypt-arithmetic problems were successfully solved using Prolog, demonstrating the language's
capabilities in handling constraint satisfaction problems. Each problem was approached by defining
the constraints and using Prolog's built-in constraint solving capabilities to find valid digit assignments.
The solutions obtained were verified for correctness, ensuring that each letter corresponded to a unique digit
and that the arithmetic equations held true. This exercise highlighted the effectiveness of Prolog in
solving complex logical problems and reinforced the understanding of constraint satisfaction techniques.

\end{document}
