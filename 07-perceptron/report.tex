\documentclass[12pt,a4paper]{report}

% ----------------------- PREAMBLES -------------------------
% ==============================================================================================
% PACKAGE IMPORTS
% ----------------------------------------------------------------------------------------------
\usepackage[margin=1in]{geometry}  % Set page margins
\usepackage{graphicx}   % Enable including images
\usepackage{titlesec}  % Required for custom section titles
\usepackage{listings}   % For including code snippits
\usepackage{xcolor}      % load the package for colors
% ===============================================================================================

% ==============================================================================================
% CONFIGURATIONS
% ----------------------------------------------------------------------------------------------
% -------------------------- LISTING CONFIGURATIONS ------------------
% Custom Listing Style
\lstset{
    basicstyle=\ttfamily\normalsize, % Monospaced font
    commentstyle=\color{green!60!black}\itshape,  % Comments in green italics
    keywordstyle=\color{blue}, % Keywords in blue
    stringstyle=\color{red}, % Strings in red
    numbers=left,  % Line numbers
    numberstyle=\scriptsize,
    stepnumber=1,
    numbersep=5pt,
    frame=single,  % Box around code
    breaklines=true, % Automatic line breaks
    showstringspaces=false,
    tabsize=4,
    captionpos=b,  % Captions below code
    abovecaptionskip=0pt,
    belowskip=5pt
}

% % Change the listing name
% \renewcommand{\lstlistingname}{Code} % Changes "Listing" to "Code"

% ------------------ GRAPHICS CONFIGURATIONS ------------------------
\graphicspath{{../images/}}  % Default folder(s) for images

% -------------------- PARAGRAPH FORMATTING --------------------------
\setlength{\parindent}{0cm}      % Size of paragraph indentation
\setlength{\parskip}{12pt}       % Vertical space between paragraphs

% ------------------ SECTION FORMATTING -----------------------------
% Format section titles to be ALL CAPS
\titleformat{\section}[block]
  {\normalfont\bfseries}
  {\thesection}{0.5cm}{\MakeUppercase}

% Customize section title spacing
\titlespacing{\section}{0pt}{5pt}{0pt}

% Suppress numbering globally for sections
\setcounter{secnumdepth}{0}
% ===============================================================================================

% --------------------- LAB INFORMATION ---------------------------
\newcommand{\PracticalDate}{23rd Janaury 2026}
\newcommand{\SubmissionDate}{1st February 2026}

\newcommand{\LabNo}{07}
\newcommand{\LabTitle}{Perceptron}

% --------------------- LAB REPORT --------------------------------
\begin{document}
% Cover Page
% ==============================================================================================
% COVER PAGE INFORMATION (MACROS)
% ----------------------------------------------------------------------------------------------
% -------------- COLLEGE INFORMATION ------------------
\newcommand{\CollegeName}{Cosmos College of \\[12pt] Management \& Technology}
\newcommand{\Affiliation}{(Affiliated to Pokhara University)}
\newcommand{\CollegeAddress}{Sitapaila, Kathmandu}

% -------------- STUDENT INFORMATION ----------------------
\newcommand{\StudentName}{Kushal Prasad Joshi}
\newcommand{\RollNo}{230345}
\newcommand{\Faculty}{Science and Technology}
\newcommand{\Department}{ICT}
\newcommand{\Group}{B}

% ----------------- LAB INFORMATION -------------------
\newcommand{\Course}{Artifical Intelligence (CMP 346)}
\newcommand{\LabInstructor}{Mr.\ Ranjan Raj Aryal}
% ==============================================================================================

% ==============================================================================================
% COVER PAGE FORMAT
% ----------------------------------------------------------------------------------------------
\begin{titlepage}
    \centering
    \vspace*{\fill}

    % ---------------------- COLLEGE INFO ------------------------
    {\Huge \textbf{\CollegeName}}\\[12pt]
    {\huge \Affiliation}\\[12pt]
    {\LARGE \CollegeAddress}
    \vspace{2.5cm}

    % ------------------------ COLLEGE LOGO ------------------------
    \includegraphics[width=0.3\textwidth]{cosmos-logo.png}
    \vspace{2.5cm}

    % ------------------------ LAB INFO -------------------------------
    {\Large \textbf{\Course\ Lab Report}}\\[10pt]
    \textbf{LAB REPORT NO:} \LabNo\\[6pt]
    \textbf{LAB REPORT ON:}\\ \MakeUppercase{\LabTitle}
    \vspace{2.5cm}

    % ----------------------------- SUBMISSION TABLE -------------------
    \begin{tabular}{p{8cm} p{8cm}}
        \textbf{SUBMITTED BY:-} & \textbf{SUBMITTED TO:-}\\[10pt]
        \textbf{Name:} \StudentName& \textbf{Department:} \Department\\[6pt]
        \textbf{Roll No:} \RollNo& \textbf{Lab Instructor:} \LabInstructor\\[6pt]
        \textbf{Faculty: }\Faculty& \textbf{Date of Experiment:} \PracticalDate\\[6pt]
        \textbf{Group:} \Group& \textbf{Date of Submission:} \SubmissionDate\\
    \end{tabular}   
\end{titlepage}

% Title
\section{Title}
\MakeUppercase{\LabTitle}

% Define Objectives
\section{Objectives}
\begin{enumerate}
    \item To understand the basic concepts of Perceptron.
    \item To implement single layer perceptron.
    \item To implement multi layer perceptron.
\end{enumerate}

% Define Requirements
\section{Requirements}
\begin{itemize}
    \item Python
    \item VS Code
    \item Libraries: NumPy
\end{itemize}

% Theory
\section{Theory}

    \textbf{Perceptron:}
    A perceptron is the simplest artificial neural network model inspired by a biological neuron. 
    It takes weighted inputs, adds a bias, and passes the result through an activation function to 
    produce an output. It is mainly used for binary classification problems.

    \textbf{Single Layer Perceptron (SLP):}
    A Single Layer Perceptron consists of an input layer directly connected to an output layer 
    without any hidden layers. It can solve only linearly separable problems such as AND and OR 
    logic gates. Learning is performed using a simple weight update rule.

    \textbf{Multi-Layer Perceptron (MLP):}
    A Multi-Layer Perceptron contains one or more hidden layers between the input and output 
    layers. It uses nonlinear activation functions and is trained using the backpropagation 
    algorithm. MLPs are capable of solving complex, non-linearly separable problems such as XOR.\

    \textbf{Epoch:}
    An epoch represents one complete pass of the entire training dataset through the neural 
    network. During each epoch, the network updates its weights to reduce the error. Multiple 
    epochs are usually required for effective learning.

    \textbf{Learning Rate:}
    The learning rate controls the magnitude of weight updates during training. A smaller 
    learning rate results in slow but stable learning, while a larger learning rate speeds 
    up training but may cause instability. It is an important hyperparameter in neural networks.

    \textbf{Weights:}
    Weights are numerical parameters associated with each input that determine its influence on 
    the output. During training, weights are adjusted to minimize the error between predicted 
    and actual outputs. Proper weight tuning improves model accuracy.

    \textbf{Bias:}
    Bias is an additional parameter added to the weighted sum of inputs. It allows the 
    activation function to be shifted, enabling the model to fit the data more flexibly. 
    Bias helps the decision boundary avoid passing through the origin.

    \textbf{Activation Function:}
    An activation function introduces non-linearity into the neural network and determines 
    whether a neuron should be activated. Common activation functions include the step function, 
    sigmoid function, and ReLU.\ Non-linearity allows networks to learn complex patterns.

    \textbf{Linearly Separable Problem:}
    A problem is said to be linearly separable if a single straight line (or hyperplane) can 
    separate the data points into different classes. Examples include AND and OR logic gates. 
    Single Layer Perceptrons can solve such problems.

    \textbf{Non-Linearly Separable Problem:}
    A non-linearly separable problem cannot be separated using a single straight line. The XOR 
    problem is a classic example. Solving such problems requires hidden layers and nonlinear 
    activation functions.

    \textbf{Backpropagation:}
    Backpropagation is a supervised learning algorithm used in Multi-Layer Perceptrons. It 
    calculates the error at the output layer and propagates it backward to update the weights. 
    This process enables the network to learn complex relationships.

    \textbf{Training Data:}
    Training data consists of input-output pairs used to train the neural network. The model 
    learns by adjusting its parameters based on this data. High-quality and sufficient training 
    data leads to better performance.

\section{Implementation}

% Single Layer Perceptron Implementation
\begin{lstlisting}[language=Python,  caption={Single Layer Perceptron Implementation}, label=listing:single-layer-perceptron]
# Implement a perceptron to learn the AND logic gate. 
# ( AND gate is Linearly separable and  Perfect for single layer perceptron)

import numpy as np

# Training data for AND gate
X = np.array([
    [0, 0],
    [0, 1],
    [1, 0],
    [1, 1]
])

y = np.array([0, 0, 0, 1])  # AND output

# Initialize weights and bias
weights = np.zeros(2)
bias = 0
learning_rate = 0.1

# Activation function
def step_function(x):
    return 1 if x >= 0 else 0

# Training
for epoch in range(10):
    for i in range(len(X)):
        linear_output = np.dot(X[i], weights) + bias
        y_pred = step_function(linear_output)
        error = y[i] - y_pred

        # Update rule
        weights += learning_rate * error * X[i]
        bias += learning_rate * error

print("Trained weights:", weights)
print("Trained bias:", bias)

# Testing
print("\nTesting AND gate:")
for i in range(len(X)):
    output = step_function(np.dot(X[i], weights) + bias)
    print(X[i], "->", output)

\end{lstlisting}

\begin{lstlisting}[caption={Output of the Single Layer Perceptron Implementation}]
Trained weights: [0.2 0.1]
Trained bias: -0.20000000000000004

Testing AND gate:
[0 0] -> 0
[0 1] -> 0
[1 0] -> 0
[1 1] -> 1
\end{lstlisting}

% Multi Layer Perceptron Implementation
\begin{lstlisting}[language=Python,  caption={Multi Layer Perceptron Implementation}, label=listing:multi-layer-perceptron]
# Implement an MLP to learn the XOR logic gate. 
# (XOR gate is for Not linearly separable and it requires hidden layer)

import numpy as np

# XOR dataset
X = np.array([
    [0, 0],
    [0, 1],
    [1, 0],
    [1, 1]
])
y = np.array([[0], [1], [1], [0]])

# Initialize weights
np.random.seed(1)
W1 = np.random.randn(2, 2)
b1 = np.zeros((1, 2))
W2 = np.random.randn(2, 1)
b2 = np.zeros((1, 1))

learning_rate = 0.1

# Activation functions
def sigmoid(x):
    return 1 / (1 + np.exp(-x))

def sigmoid_derivative(x):
    return x * (1 - x)

# Training
for epoch in range(5000):
    # Forward pass
    hidden_input = np.dot(X, W1) + b1
    hidden_output = sigmoid(hidden_input)

    output_input = np.dot(hidden_output, W2) + b2
    y_pred = sigmoid(output_input)

    # Backpropagation
    error = y - y_pred
    d_output = error * sigmoid_derivative(y_pred)

    d_hidden = d_output.dot(W2.T) * sigmoid_derivative(hidden_output)

    # Update weights
    W2 += hidden_output.T.dot(d_output) * learning_rate
    b2 += np.sum(d_output, axis=0, keepdims=True) * learning_rate
    W1 += X.T.dot(d_hidden) * learning_rate
    b1 += np.sum(d_hidden, axis=0, keepdims=True) * learning_rate

# Testing
print("\nTesting XOR gate:")
for i in range(len(X)):
    hidden = sigmoid(np.dot(X[i], W1) + b1)
    output = sigmoid(np.dot(hidden, W2) + b2)
    print(X[i], "->", round(output.item()))

\end{lstlisting}

\begin{lstlisting}[caption={Output of the Multi Layer Perceptron Implementation}]
Testing XOR gate:
[0 0] -> 0
[0 1] -> 1
[1 0] -> 1
[1 1] -> 0
\end{lstlisting}

% Conclusion
\section{Conclusion}
In this lab, we explored the concepts of Single Layer Perceptron (SLP) and Multi-Layer Perceptron (MLP).
We implemented an SLP to learn the AND logic gate, which is linearly separable, and successfully trained it to produce correct outputs.
We also implemented an MLP to learn the XOR logic gate, which is not linearly separable, and demonstrated its ability to classify the inputs correctly after training.
These implementations highlight the capabilities of perceptrons in solving different types of classification problems.

\end{document}