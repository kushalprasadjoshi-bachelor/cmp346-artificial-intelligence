\documentclass[12pt,a4paper]{report}

% ----------------------- PREAMBLES -------------------------
% ==============================================================================================
% PACKAGE IMPORTS
% ----------------------------------------------------------------------------------------------
\usepackage[margin=1in]{geometry}  % Set page margins
\usepackage{graphicx}   % Enable including images
\usepackage{titlesec}  % Required for custom section titles
\usepackage{listings}   % For including code snippits
\usepackage{xcolor}      % load the package for colors
% ===============================================================================================

% ==============================================================================================
% CONFIGURATIONS
% ----------------------------------------------------------------------------------------------
% -------------------------- LISTING CONFIGURATIONS ------------------
% Custom Listing Style
\lstset{
    basicstyle=\ttfamily\normalsize, % Monospaced font
    commentstyle=\color{green!60!black}\itshape,  % Comments in green italics
    keywordstyle=\color{blue}, % Keywords in blue
    stringstyle=\color{red}, % Strings in red
    numbers=left,  % Line numbers
    numberstyle=\scriptsize,
    stepnumber=1,
    numbersep=5pt,
    frame=single,  % Box around code
    breaklines=true, % Automatic line breaks
    showstringspaces=false,
    tabsize=4,
    captionpos=b,  % Captions below code
    abovecaptionskip=0pt,
    belowskip=5pt
}

% % Change the listing name
% \renewcommand{\lstlistingname}{Code} % Changes "Listing" to "Code"

% ------------------ GRAPHICS CONFIGURATIONS ------------------------
\graphicspath{{../images/}}  % Default folder(s) for images

% -------------------- PARAGRAPH FORMATTING --------------------------
\setlength{\parindent}{0cm}      % Size of paragraph indentation
\setlength{\parskip}{12pt}       % Vertical space between paragraphs

% ------------------ SECTION FORMATTING -----------------------------
% Format section titles to be ALL CAPS
\titleformat{\section}[block]
  {\normalfont\bfseries}
  {\thesection}{0.5cm}{\MakeUppercase}

% Customize section title spacing
\titlespacing{\section}{0pt}{5pt}{0pt}

% Suppress numbering globally for sections
\setcounter{secnumdepth}{0}
% ===============================================================================================

% --------------------- LAB INFORMATION ---------------------------
\newcommand{\PracticalDate}{18th Janaury 2026}
\newcommand{\SubmissionDate}{23rd Janaury 2026}

\newcommand{\LabNo}{06}
\newcommand{\LabTitle}{Pattern Recognition}

% --------------------- LAB REPORT --------------------------------
\begin{document}
% Cover Page
% ==============================================================================================
% COVER PAGE INFORMATION (MACROS)
% ----------------------------------------------------------------------------------------------
% -------------- COLLEGE INFORMATION ------------------
\newcommand{\CollegeName}{Cosmos College of \\[12pt] Management \& Technology}
\newcommand{\Affiliation}{(Affiliated to Pokhara University)}
\newcommand{\CollegeAddress}{Sitapaila, Kathmandu}

% -------------- STUDENT INFORMATION ----------------------
\newcommand{\StudentName}{Kushal Prasad Joshi}
\newcommand{\RollNo}{230345}
\newcommand{\Faculty}{Science and Technology}
\newcommand{\Department}{ICT}
\newcommand{\Group}{B}

% ----------------- LAB INFORMATION -------------------
\newcommand{\Course}{Artifical Intelligence (CMP 346)}
\newcommand{\LabInstructor}{Mr.\ Ranjan Raj Aryal}
% ==============================================================================================

% ==============================================================================================
% COVER PAGE FORMAT
% ----------------------------------------------------------------------------------------------
\begin{titlepage}
    \centering
    \vspace*{\fill}

    % ---------------------- COLLEGE INFO ------------------------
    {\Huge \textbf{\CollegeName}}\\[12pt]
    {\huge \Affiliation}\\[12pt]
    {\LARGE \CollegeAddress}
    \vspace{2.5cm}

    % ------------------------ COLLEGE LOGO ------------------------
    \includegraphics[width=0.3\textwidth]{cosmos-logo.png}
    \vspace{2.5cm}

    % ------------------------ LAB INFO -------------------------------
    {\Large \textbf{\Course\ Lab Report}}\\[10pt]
    \textbf{LAB REPORT NO:} \LabNo\\[6pt]
    \textbf{LAB REPORT ON:}\\ \MakeUppercase{\LabTitle}
    \vspace{2.5cm}

    % ----------------------------- SUBMISSION TABLE -------------------
    \begin{tabular}{p{8cm} p{8cm}}
        \textbf{SUBMITTED BY:-} & \textbf{SUBMITTED TO:-}\\[10pt]
        \textbf{Name:} \StudentName& \textbf{Department:} \Department\\[6pt]
        \textbf{Roll No:} \RollNo& \textbf{Lab Instructor:} \LabInstructor\\[6pt]
        \textbf{Faculty: }\Faculty& \textbf{Date of Experiment:} \PracticalDate\\[6pt]
        \textbf{Group:} \Group& \textbf{Date of Submission:} \SubmissionDate\\
    \end{tabular}   
\end{titlepage}

% Title
\section{Title}
\MakeUppercase{\LabTitle}

% Define Objectives
\section{Objectives}
\begin{enumerate}
    \item To understand the basic concepts of pattern recognition.
    \item To implement KNN algorithm.
    \item To implement K-Means Clustering algorithm.
\end{enumerate}

% Define Requirements
\section{Requirements}
\begin{itemize}
    \item Python
    \item VS Code
    \item Libraries: OpenCV, NumPy, Matplotlib
\end{itemize}

% Theory
\section{Theory}
Pattern recognition is a branch of machine learning that focuses on the identification and
classification of patterns in data. It involves the use of algorithms and statistical methods
to recognize patterns and make predictions based on the identified patterns.

\textbf{K-Nearest Neighbors (kNN):}
kNN is one of the simplest of classification algorithms available for supervised learning. The idea is to 
search for closest match of the test data in feature space. We will look into it with below image. 

\begin{figure}
    \centering
    \includegraphics[width=0.8\textwidth]{knn.png}
    \caption{K-Nearest Neighbors}
\end{figure}

In the image, there are two families, Blue Squares and Red Triangles. We call each family 
as Class. Their houses are shown in their town map which we call feature space. (You can consider a 
feature space as a space where all data are projected. For example, consider a 2D coordinate space. 
Each data has two features, x and y coordinates. You can represent this data in your 2D coordinate 
space. Now imagine if there are three features, you need 3D space. Now consider N features, where you 
need N-dimensional space, This N-dimensional space is its feature space. In this image, you can consider 
it as a 2D case with two features). 

Now a new member comes into the town and creates a new home, which is shown as green 
circle. He should be added to one of these Blue/Red families. We call that process, Classification now, let 
us apply this algorithm, kNN.\

One method is to check who is his nearest neighbor. From the image, it is clear it is the Red 
Triangle family. So he is also added into Red Triangle. This method is called simply Nearest Neighbor, 
because classification depends only on the nearest neighbor.\

\textbf{K-Means Clustering:} 
K-Means Clustering is an unsupervised machine learning algorithm used to group similar data points
into clusters based on their features. The algorithm works by iteratively assigning data points to
the nearest cluster centroid and updating the centroids based on the assigned points until convergence.

% Implementation
\section{Implementation}

% Question 1: KNN Implementation
\begin{lstlisting}[language=Python,  caption={K-Nearest Neighbors Implementation}, label=listing:knn]
import cv2 
import numpy as np 
import matplotlib.pyplot as plt 

# Feature set containing (x,y) values of 25 known/training data 
trainData = np.random.randint(0, 100, (25, 2)).astype(np.float32) 

# Labels each one either Red or Blue with numbers 0 and 1 
responses = np.random.randint(0, 2, (25, 1)).astype(np.float32) 

# Take Red families and plot them 
red = trainData[responses.ravel() == 0] 
plt.scatter(red[:, 0], red[:, 1], 80, c='r', marker='^') 

# Take Blue families and plot them 
blue = trainData[responses.ravel() == 1] 
plt.scatter(blue[:, 0], blue[:, 1], 80, c='b', marker='s') 
plt.show() 
\end{lstlisting}

\begin{figure}
    \centering
    \includegraphics[width=0.8\textwidth]{output-knn.png}
    \caption{Output of K-Nearest Neighbors Implementation}
\end{figure}

% Question 2: K-Means Clustering Implementation
\begin{lstlisting}[language=Python,  caption={K-Means Clustering Implementation}, label=listing:kmeans]
import numpy as np 
import cv2 
from matplotlib import pyplot as plt 

X = np.random.randint(25,50,(25,2)) 
Y = np.random.randint(60,85,(25,2)) 
Z = np.vstack((X,Y)) 

# convert to np.float32 
Z = np.float32(Z) 

# define criteria and apply kmeans() 
criteria = (cv2.TERM_CRITERIA_EPS + cv2.TERM_CRITERIA_MAX_ITER, 10, 1.0) 
ret,label,center=cv2.kmeans(Z,2,None,criteria,10,cv2.KMEANS_RANDOM_CENTERS) 

# Now separate the data, Note the flatten() 
A = Z[label.ravel()==0] 
B = Z[label.ravel()==1] 

# Plot the data 
plt.scatter(A[:,0],A[:,1]) 
plt.scatter(B[:,0],B[:,1],c = 'r') 
plt.scatter(center[:,0],center[:,1],s = 80,c = 'y', marker = 's') 
plt.xlabel('Height'),plt.ylabel('Weight') 
plt.show() 
\end{lstlisting}

\begin{figure}
    \centering
    \includegraphics[width=0.8\textwidth]{output-k-means-clustering.png}
    \caption{Output of K-Means Clustering Implementation}
\end{figure}

% Question 3: K-Means Clustering Problem

\begin{lstlisting}[language=Python,  caption={K-Means Clustering Implementation}, label=listing:kmeans]
"""
Question:
Solve the problem using K-Means Clustering method and analyze the results.

Given:
- Generate two sets of random data:
  x in the range [25, 100]
  y in the range [175, 255]
- Combine them into a single dataset
- Plot the histogram
- Apply K-Means clustering
- Analyze the clustering result
"""

# ------- IMPORT REQUIRED LIBRARIES --------
import numpy as np
import cv2
from matplotlib import pyplot as plt

# ---------- DATA GENERATION ----------
# Generate 25 random values between 25 and 100
x = np.random.randint(25, 100, 25)

# Generate 25 random values between 175 and 255
y = np.random.randint(175, 255, 25)

# Combine both arrays into a single array
z = np.hstack((x, y))

# Reshape data into a column vector (required by OpenCV K-Means)
z = z.reshape((50, 1))

# Convert data type to float32 (required by OpenCV)
z = np.float32(z)

# ----- HISTOGRAM OF ORIGINAL DATA ------
plt.hist(z, 256, [0, 256])
plt.title("Histogram of Original Data")
plt.xlabel("Value")
plt.ylabel("Frequency")
plt.show()

# ------- APPLY K-MEANS CLUSTERING ------
# Define stopping criteria:
# Stop after 10 iterations or when accuracy reaches epsilon = 1.0
criteria = (cv2.TERM_CRITERIA_EPS + cv2.TERM_CRITERIA_MAX_ITER,
            10, 1.0)

# Number of clusters (since data has two distinct groups)
K = 2

# Apply K-Means clustering
ret, label, center = cv2.kmeans(
    z,                      # input data
    K,                      # number of clusters
    None,                   # initial labels
    criteria,               # termination criteria
    10,                     # number of attempts
    cv2.KMEANS_RANDOM_CENTERS
)

# ------ DISPLAY CLUSTER CENTERS -------
print("Cluster Centers:")
print(center)

# ------- SEPARATE DATA BASED ON CLUSTERS ------
cluster1 = z[label.ravel() == 0]
cluster2 = z[label.ravel() == 1]

# ------- PLOT CLUSTERED DATA --------
plt.hist(cluster1, 256, [0, 256], alpha=0.6, label="Cluster 1")
plt.hist(cluster2, 256, [0, 256], alpha=0.6, label="Cluster 2")

# Plot cluster centers as vertical lines
plt.axvline(center[0], color='black', linestyle='--', label='Center 1')
plt.axvline(center[1], color='red', linestyle='--', label='Center 2')

plt.title("K-Means Clustering Result (K = 2)")
plt.xlabel("Value")
plt.ylabel("Frequency")
plt.legend()
plt.show()

# ------ RESULT ANALYSIS -------
"""
Analysis:
- The histogram of the original data shows two distinct peaks.
- K-Means with K = 2 successfully separates the data into two clusters.
- One cluster corresponds to low-range values (25-100).
- The other cluster corresponds to high-range values (175-255).
- Cluster centers represent the mean value of each group.
- Since the data is well separated, K-Means converges quickly and accurately.
"""

print("K-Means clustering successfully divided the data into two distinct clusters.")

\end{lstlisting}

\begin{figure}
    \centering
    \includegraphics[width=0.8\textwidth]{output-k-means-clustering-question.png}
    \caption{Output of K-Means Clustering Question}
\end{figure}

\begin{figure}
    \centering
    \includegraphics[width=0.8\textwidth]{output-k-means-clustering-solution.png}
    \caption{Output of K-Means Clustering Solution Histogram}
\end{figure}

\begin{lstlisting}[caption={Output of the K-Means Clustering Problem Implementation}]
K-Means clustering successfully divided the data into two distinct clusters.
\end{lstlisting}

% Results and Conclusion
\section{Results and Conclusion}
In this lab, we explored the concepts of pattern recognition through the implementation of K-Nearest Neighbors (kNN)
and K-Means Clustering algorithms. The kNN algorithm effectively classified data points based on their
proximity to known classes, while the K-Means Clustering algorithm successfully grouped similar data points into distinct clusters.
The implementations demonstrated the practical applications of these algorithms in pattern recognition tasks.

\end{document}